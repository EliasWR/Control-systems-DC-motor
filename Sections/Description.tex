\section{Beskrivelse}

Dette er en besvarelse på arbeidskrav 1, Reguleringsteknikk 2002. Oppgaven er besvart av studenter ved NTNU Ålesund, Oktober 2022.

Vi skal benytte en ankerstyrt likestrømmotor som en posisjonsservo, det vil si at motorakslingen skal regulere seg inn på en gitt vinkel. Vi starter med å stille opp en dynamisk modell av motoren. Motoren består av en elektrisk del (ankerkretsen) og en mekanisk del (momentbalanse for akslingen).

Motoren har følgende data:
\begin{center}































	\begin{tabular}{ll}
		Ankerresistans:        & $R_a = 10\ohm$    \\
		Ankerinduktans:        & $L_a = 1H$        \\
		Ankerstrøm:            & $i_a$             \\
		Motorkonstant:         & $K_e = K_T = 1$   \\
		Treghetsmoment:        & $J_m = 0.01kgm^2$ \\
		Friksjon:              & $b = 0.001Nms$    \\
		Mot-indusert spenning: & $e_m = K_e\omega$ \\
		Motormoment:           & $T = K_ti_a$      \\
		Akselvinkel:           & $\theta$          \\
		Vinkelhastighet        & $\omega$          \\
	\end{tabular}
\end{center}